\documentclass{beamer}

\title{PageRank}
\author{Ben Burns, Dan Magazu, Lucas Chagas, \\Thomas Webster, Trung Do}
\institute{MATH 455}
\date{Fall 2021}

\usepackage{outlines}
\usepackage{graphicx}
\graphicspath{{./images}}

\begin{document}
\frame{\titlepage}
\begin{frame}[t]
\frametitle{Motivation}
\begin{itemize}
    \setlength\itemsep{0.5em}
    \item Problem: the internet has a lot of information. 
    \item A lot of the information out there either isn't relevant to us, or is incorrect.
    \item Motivation: we want a program that, when provided a phrase, returns webpages with information relevant to the input
    \item Intuitive solution: return back websites that contain that phrase in content of the pages
    \item This helps us find more \emph{relevant} pages, but we can't know of we're getting the best information, much less accurate information, without manually going through each result
    \item Want a stronger solution
\end{itemize}
\end{frame}

\begin{frame}[t]
\frametitle{PageRank}
\begin{outline}
    \1 Invented by Sergey Brin and Larry Page (1998)\footnotemark 
        \2 Publication marks them becoming co-founders of Google  
    \1 Idea: we want some way to numerically score each webpage based on how "important" it is
    \1 Algorithm scores each page based on 
        \2 How many other pages link to it
        \2 The importance of each of its citations 
    \1 We can then numerically order pages to rank them (their "PageRank")
    \1 PageRank: the procedure for scoring each website
    \1 Google: the database that indexes the PageRank of each website for use
\end{outline}
\footnotetext[1]{Many use the year of the original manuscript, 1996}
\end{frame}
\begin{frame}[t]
\frametitle{Formalizing the PageRank problem}
\begin{outline}
    \1 We're going to construct a directed graph $G = (V, E)$
    \1 For each website we consider, we construct a node $v_i \in V$
    \1 For two distinct nodes $v_i,\ v_j \in V$, the \emph{directed} edge $v_iv_j \in E$ iff there is a link on website $i$ that goes to website $j$.
    \1 If $v_i$ and $v_j$ are not distinct (a website is linking to itself), we ignore the link and do not construct a loop edge.
        \2 $G$ is not a psuedograph
    \1 Multiple hyperlinks on page $i$ to page $j$ are all represented by the single, directed edge 
        \2 $G$ is not a multigraph.
\end{outline}
\end{frame}

\begin{frame}
\frametitle{Visual Representation}
\begin{center}
    \includegraphics[width=0.6\textwidth]{unweighted.png}
\end{center}
\begin{outline}
    \1 We have a set of 4 websites
    \1 Each edge represents a hyperlink from the origin node to the destinationx node 
\end{outline}
\end{frame}

\begin{frame}
\frametitle{Adjacency Matrix}
\begin{columns}
    \begin{column}{0.5\textwidth}
        \centering
        \includegraphics[width=\textwidth]{unweighted.png}
    \end{column}
    \begin{column}{0.5\textwidth}
        \centering
        {\Large$A = \begin{pmatrix}
            0 & 1 & 1 & 1\\
            0 & 0 & 1 & 1\\
            1 & 0 & 0 & 0\\
            1 & 0 & 1 & 0\\
        \end{pmatrix}$}
    \end{column}
\end{columns}
\begin{outline}
    \1 No self loops means the main diagonal is all zeros
\end{outline}
\end{frame}
\begin{frame}[t]
\frametitle{Applying PageRank Values and Edge Weights}
\begin{outline}
    \1 At first, we assign 
    \1 For each \emph{vertex} $v_i$, we assign some PageRank value $PR(v_i)$    
        \2 TODO: How to calculate?
    \1 Consider the set of vertices that $v_i$ has an edge to 
        \2 $V_i = \{v_j | v_iv_j \in E\}$
    \1 For each vertex in the set, the weight of the edge to that vertex will be the PageRank value of the source node divided by the number of outbound edges it has
        \2 $\forall v_j \in V_i: w(v_iv_j) = \dfrac{PR(v_i)}{|V_i|}$
    \1 In other words, all the outbound edges from a particular vertex will have the same weight
\end{outline}
\end{frame}

\begin{frame}
\frametitle{Visual Representation}
\begin{center}
    \includegraphics[width=0.6\textwidth]{weighted.png}
\end{center}
\begin{outline}
    \1 In this case, all nodes have a PageRank value of 1.
\end{outline}
\end{frame}

\begin{frame}
\frametitle{Transition Matrix}
\begin{columns}
    \begin{column}{0.5\textwidth}
        \centering
        \includegraphics[width=\textwidth]{weighted.png}
    \end{column}
    \begin{column}{0.5\textwidth}
        \centering
        $T = \begin{pmatrix}
            0 & 1/3 & 1/3 & 1/3\\
            0 & 0 & 1/2 & 1/2\\
            1 & 0 & 0 & 0\\
            1/2 & 0 & 1/2 & 0\\
        \end{pmatrix}$
    \end{column}
\end{columns}
\begin{outline}
    \1 All entries of the transition matrix are non-negative
    \1 If $v_i$ has at least one outgoing edge, the sum of the entries in row $i$ is 1
        \2 Else, the sum is 0.
    \1 This is a \emph{column stochastic matrix}
\end{outline}
\end{frame}

\end{document}